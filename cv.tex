% LaTeX resume using res.cls
\documentclass[line,margin]{res} 
\usepackage{cite}
\usepackage{url}
%\usepackage{helvetica} % uses helvetica postscript font (download helvetica.sty)
%\usepackage{newcent}   % uses new century schoolbook postscript font 

%Mess with margins
%Should be of the form:
%
%\addtolength{\textwidth}{ain}
%\addtolength{\hoffset}{-bin}
%\addtolength{\textheight}{cin}
%\addtolength{\voffset}{-cin}
%
%where a = 2b

% \addtolength{\textheight}{0.2in}
% \addtolength{\voffset}{-0.2in}

% Want to change how my name appears
% \renewcommand{\namefont}{\large\bf}
\renewcommand{\namefont}{\huge\textsc}

\begin{document}

\name{Daniel Ritchie}
% \address can be used twice to have two lines of address
% \address{\url{dritchie@stanford.edu} | \url{stanford.edu/~dritchie}}
\address{\url{dritchie@stanford.edu} $\cdot$ (530) 409-6656}
\address{353 Serra Mall \#381, Stanford, CA 94305}
 
\begin{resume}

\section{RESEARCH INTERESTS}
In my research, I use probabilistic programming to solve procedural modeling and design problems. Developing detailed, interactive procedural content for computer graphics applications requires authoring complex generative processes (e.g. a procedure that generates residential buildings) that can also satisfy functional and aesthetic constraints (e.g. generated buildings should be structurally stable). Bayesian probabilistic inference provides one way to achieve these goals: the generative process serves as a prior distribution, and constraints are encoded as a likelihood function. Probabilistic programs provide a universal, human-understandable representation for such probabilistic models. 

Inference on such general models is often intractable, relying on slowly-converging approximate methods. In my work, I develop new inference algorithms and improve implementations of existing algorithms to accelerate inference for the complex programs found in graphics applications. I wrote \textbf{Quicksand}, a low-level probabilistic programming system that generates fast machine code for Metropolis-Hastings and Hamiltonian Monte Carlo inference, providing faster inference for procedural models with tight constraints between multiple variables. I created the \textbf{Stochastically-Ordered Sequential Monte Carlo} algorithm, which extends Sequential Monte Carlo to recursively branching programs and converges faster than Metropolis-Hastings for recursive procedural modeling programs. I also designed \textbf{C3}, a lightweight system for speeding up Metropolis-Hastings proposals by incrementalizing away redundant computation. While developed with procedural modeling in mind, these are all general-purpose tools of broader utility to the probabilistic programming community at large.
 
\section{EDUCATION}
\textbf{Stanford University} \\
PhD, Computer Science \\
Advisor: Pat Hanrahan \\
Expected completion date: June 2016 \\
\\
\textbf{Stanford University} \\
MS, Computer Science \\
Conferred April 2013 \\
\\
\textbf{University of California Berkeley} \\
BA, Computer Science \\
Conferred May 2010 \\


\section{REFEREED PUBLICATIONS}
\textbf{C3: Lightweight Incrementalized MCMC for Probabilistic Programs using Continuations and Callsite Caching}. 
Daniel Ritchie, Andreas Stuhlm\"uller, Noah D. Goodman. 
In submission to \emph{NIPS 2015}.
\\ \\
\textbf{Controlling Procedural Modeling Programs with Stochastically-Ordered Sequential Monte Carlo}. 
Daniel Ritchie, Ben Mildenhall, Noah D. Goodman, and Pat Hanrahan. 
In \emph{The Proceedings of SIGGRAPH 2015}.
\\ \\
\textbf{Generating Design Suggestions under Tight Constraints with Gradient-based Probabilistic Programming}. 
Daniel Ritchie, Sharon Lin, Noah D. Goodman, and Pat Hanrahan. 
In \emph{The Proceedings of Eurographics 2015}. \textbf{Best Paper Honorable Mention}
\\ \\
\textbf{Quicksand: A Lightweight Embedding of Probabilistic Programming for Procedural Modeling and Design}. 
Daniel Ritchie. 
In \emph{The 3rd NIPS Workshop on Probabilistic Programming, 2014}.
\\ \\
\textbf{First-class Runtime Generation of High-performance Types using Exotypes}. 
Zach Devito, Daniel Ritchie, Matthew Fisher, Alex Aiken, and Pat Hanrahan. 
In \emph{The Proceedings of PLDI 2014}.
\\ \\
\textbf{Probabilistic Color-by-Numbers: Suggesting Pattern Colorizations Using Factor Graphs}. 
Sharon Lin, Daniel Ritchie, Matthew Fisher, and Pat Hanrahan. 
In \emph{The Proceedings of SIGGRAPH 2013}.
\\ \\
\textbf{Example-based Synthesis of 3D Object Arrangements}. 
Matthew Fisher, Daniel Ritchie, Manolis Savva, Thomas Funkhouser, and Pat Hanrahan. 
In \emph{The Proceedings of SIGGRAPH Asia 2012}.
\\ \\
\textbf{d.tour: Style-based Exploration of Design Example Galleries}. 
Daniel Ritchie, Ankita Arvind Kejriwal, and Scott R. Klemmer. 
In \emph{The Proceedings of UIST 2011}.
\\ \\
\textbf{Dynamic Local Remeshing for Elastoplastic Simulation}. 
Martin Wicke, Daniel Ritchie, Bryan M. Klingner, Sebastian Burke, Jonathan R. Shewchuk, and James F. O'Brien. 
In \emph{The Proceedings of SIGGRAPH 2010}.
\\ \\
\textbf{Interactive Simulation of Surgical Needle Insertion and Steering}. 
Nuttapong Chentanez, Ron Alterovitz, Daniel Ritchie, Lita Cho, Kris K. Hauser, Ken Goldberg, Jonathan R. Shewchuk, and James F. O'Brien. 
In \emph{The Proceedings of SIGGRAPH 2009}. 
 

% \section{BOOK CHAPTERS}


% Control how position information is displayed
\begin{format}
\employer{l}\location{r}\\
\title{l}\dates{r}\\
\body\\
\end{format}


\section{EMPLOYMENT}

\employer{\textbf{Research Intern}}
\title{\textsc{Adobe Creative Technologies Lab}}
\dates{Summer 2011}
\location{San Francisco, CA}
\begin{position}
Mentored by Mira Dontcheva. I investigated the creative process for webcomic artists, identified the early planning/layout phase as an opportunity for computational assistance, and prototyped a tablet-based tool for quick composition and manipulation of rough thumbnail sketches. My work resulted in a patent for new methods of creating digital comics.
\end{position}

\employer{\textbf{Technical Director Intern}}
\title{\textsc{Pixar Animation Studios}}
\dates{Summer 2009}
\location{Emeryville, CA}
\begin{position}
I worked on and am credited in the short film ``Day and Night.'' I co-developed the film's pipeline for integrating 2D-animated characters into 3D scenes, and I also optimized the rendering of shots featuring hundreds of instances of desert vegetation. In addition, I wrote tools that allowed Pixar's in-house volumetric renderer to work with volumetric effects authored in Houdini.
\end{position}

\employer{\textbf{Software Intern}}
\title{\textsc{Hewlett-Packard}}
\dates{Summer 2008}
\location{Roseville, CA}
\begin{position}
I prototyped an application for user authentication and permissions management on enterprise networked print devices.
\end{position}


\section{TEACHING \& MENTORING}

% Going to abuse some commands here...

\employer{\textbf{Teaching Assistant}}
\title{\textsc{Stanford CS 348b:}}
\dates{Spring 2014}
\location{Stanford, CA}
\begin{position}
\textsc{Image Synthesis Techniques}\\
I held office hours, evaluated student work, and advised students on open-ended final projects for this graduate-level course on advanced physically-based rendering techniques.
\end{position}

\employer{\textbf{Research Mentor}}
\title{\textsc{Stanford CURIS Program}}
\dates{Summer 2013}
\location{Stanford, CA}
\begin{position}
I mentored an undergraduate student as part of the Stanford CS Undergraduate Research Internship (CURIS) program. The student, Ben Mildenhall, developed a system for `Bayesian Policy Search,' using Markov Chain Monte Carlo to infer high-reward policies for simple 2D game agents. Ben later contributed to my SIGGRAPH 2015 paper, and he is now a Ph.D. student at UC Berkeley.
\end{position}

\employer{\textbf{Teaching Assistant}}
\title{\textsc{Stanford CS 148:}}
\dates{Fall 2011}
\location{Stanford, CA}
\begin{position}
\textsc{Introduction to Computer Graphics and Imaging}\\
For this course, in addition to my office hours and grading responsibilities, I also designed a new raytracing assignment from the ground-up and gave a guest lecture on raytracing.
\end{position}

\employer{\textbf{Teaching Assistant}}
\title{\textsc{UC Berkeley CS 184:}}
\dates{Fall 2009, Spring 2010}
\location{Berkeley, CA}
\begin{position}
\textsc{Foundations of Computer Graphics}\\
I led student discussion sections for this introductory computer graphics course, where I designed small group activities, gave mini-lectures, and discussed the nuts and bolts of programming assignments.
\end{position}

\employer{\textbf{Student Facilitator}}
\title{\textsc{UCBUGG}}
% \dates{December 2008 -- May 2010}
\dates{Spring 2009, Fall 2009, Spring 2010}
\location{Berkeley, CA}
\begin{position}
For three semesters, I was part of the leadership team for the UC Berkeley Undergraduate Graphics Group (UCBUGG), an entirely student-run program that helps students learn the principles of 3D animation and guides them through the creation of their first animated short film. I designed exercises, gave lectures, and provided technical support for 3D software such as Maya.
\end{position}

\employer{\textbf{Tutor}}
\title{\textsc{UC Berkeley Self-Paced Center}}
\dates{Fall 2008}
\location{Berkeley, CA}
\begin{position}
I tutored students at UC Berkeley's Computer Science Self-Paced Center, an organization offering university-recognized courses that allow self-directed students to learn a new programming language at their own pace.
\end{position}


% \section{SERVICE}


\section{OPEN-SOURCE SOFTWARE}

\textbf{WebPPL} \\
\url{http://webppl.org} \\
I am one of the main contributors to WebPPL, a state-of-the-art probabilistic programming language that uses continuation passing style to support enumeration-based, sampling-based, and variational inference algorithms. My C3 system is available in WebPPL as the \texttt{IncrementalMH} inference method. I have also used WebPPL to implement an open-source demo of my Stochastically-Ordered Sequential Monte Carlo algorithm that runs entirely in the browser (\url{http://dritchie.github.io/web-procmod}).
\\ \\
\textbf{probabilistic-js} \\
\url{https://github.com/dritchie/probabilistic-js} \\
I am the author of probabilistic-js, a lightweight embedding of probabilistic programming in Javascript. Like WebPPL, probabilistic-js can run in-browser, and it is the inference engine behind the online course \emph{Probabilistic Models of Cognition} (\url{https://probmods.org}).
\\ \\
\textbf{Quicksand} \\
\url{http://dritchie.github.io/quicksand/} \\
I am the author of Quicksand, a lightweight embedding of probabilistic programming in the high-performance Terra language. Quicksand uses Terra's metaprogramming and code-generation constructs to compile fast machine code for MCMC inference on probabilistic programs.


\section{PATENTS}

\textbf{Methods and Apparatus for Comic Creation} (US 20130073952 A1)

\section{AWARDS \& HONORS}
Eurographics Best Paper Honorable Mention, 2015 \\
Stanford Graduate Fellowship, 2010-2015 \\ 
UC Berkeley EECS Departmental Citation, 2010 \\
UC Berkeley Computer Science Highest Achievement Award, 2010 \\
CRA Outstanding Undergraduate Researcher Honorable Mention, 2010 \\
UC Berkeley Edward Frank Kraft Scholarship, 2007 \\

% \section{OTHER ACTIVITIES}

\section{REFERENCES}

\textbf{Pat Hanrahan} \\
Canon USA Professor of Computer Science \\
Stanford University  \\
\url{hanrahan@cs.stanford.edu}
\\ \\
\textbf{Noah Goodman} \\
Assistant Professor of Psychology \\
Stanford University  \\
\url{ngoodman@stanford.edu}
\\ \\
\textbf{Thomas Funkhouser} \\
Professor of Computer Science \\
Princeton University  \\
\url{funk@cs.princeton.edu}

\end{resume}

\end{document}







