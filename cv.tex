% LaTeX resume using res.cls
\documentclass[line,margin]{res} 
\usepackage{cite}
\usepackage{url}
%\usepackage{helvetica} % uses helvetica postscript font (download helvetica.sty)
%\usepackage{newcent}   % uses new century schoolbook postscript font 

%Mess with margins
%Should be of the form:
%
%\addtolength{\textwidth}{ain}
%\addtolength{\hoffset}{-bin}
%\addtolength{\textheight}{cin}
%\addtolength{\voffset}{-cin}
%
%where a = 2b

% \addtolength{\textheight}{0.2in}
% \addtolength{\voffset}{-0.2in}

\begin{document}

\name{Daniel Ritchie}
% \address can be used twice to have two lines of address
\address{dritchie@stanford.edu | \url{stanford.edu/~dritchie}}
 
\begin{resume}

\section{RESEARCH INTERESTS}
One or two paragraphs about what I do.
 
\section{EDUCATION}
\textbf{Stanford University} \\
\emph{Ph.D. Computer Science} \\
Adviser: Pat Hanrahan \\
Expected completion date: June 2016 \\
\\
\textbf{Stanford University} \\
\emph{M.S. Computer Science} \\
Conferred April 2013 \\
\\
\textbf{University of California Berkeley} \\
\emph{B.A. Computer Science} \\
Conferred May 2010 \\


\section{REFEREED PUBLICATIONS}
\textbf{Controlling Procedural Modeling Programs with Stochastically-Ordered Sequential Monte Carlo}. 
Daniel Ritchie, Ben Mildenhall, Noah D. Goodman, and Pat Hanrahan. 
In \emph{The Proceedings of SIGGRAPH 2015}.
\\ \\
\textbf{Generating Design Suggestions under Tight Constraints with Gradient-based Probabilistic Programming}. 
Daniel Ritchie, Sharon Lin, Noah D. Goodman, and Pat Hanrahan. 
In \emph{The Proceedings of Eurographics 2015}. \textbf{Best Paper Honorable Mention}
\\ \\
\textbf{Quicksand: A Lightweight Embedding of Probabilistic Programming for Procedural Modeling and Design}. 
Daniel Ritchie. 
In \emph{The 3rd NIPS Workshop on Probabilistic Programming, 2014}.
\\ \\
\textbf{First-class Runtime Generation of High-performance Types using Exotypes}. 
Zach Devito, Daniel Ritchie, Matthew Fisher, Alex Aiken, and Pat Hanrahan. 
In \emph{The Proceedings of PLDI 2014}.
\\ \\
\textbf{Probabilistic Color-by-Numbers: Suggesting Pattern Colorizations Using Factor Graphs}. 
Sharon Lin, Daniel Ritchie, Matthew Fisher, and Pat Hanrahan. 
In \emph{The Proceedings of SIGGRAPH 2013}.
\\ \\
\textbf{Example-based Synthesis of 3D Object Arrangements}. 
Matthew Fisher, Daniel Ritchie, Manolis Savva, Thomas Funkhouser, and Pat Hanrahan. 
In \emph{The Proceedings of SIGGRAPH Asia 2012}.
\\ \\
\textbf{d.tour: Style-based Exploration of Design Example Galleries}. 
Daniel Ritchie, Ankita Arvind Kejriwal, and Scott R. Klemmer. 
In \emph{The Proceedings of UIST 2011}.
\\ \\
\textbf{Dynamic Local Remeshing for Elastoplastic Simulation}. 
Martin Wicke, Daniel Ritchie, Bryan M. Klingner, Sebastian Burke, Jonathan R. Shewchuk, and James F. O'Brien. 
In \emph{The Proceedings of SIGGRAPH 2010}.
\\ \\
\textbf{Interactive Simulation of Surgical Needle Insertion and Steering}. 
Nuttapong Chentanez, Ron Alterovitz, Daniel Ritchie, Lita Cho, Kris K. Hauser, Ken Goldberg, Jonathan R. Shewchuk, and James F. O'Brien. 
In \emph{The Proceedings of SIGGRAPH 2009}. 
 

% \section{BOOK CHAPTERS}


% Control how position information is displayed
\begin{format}
\employer{l}\location{r}\\
\title{l}\dates{r}\\
\body\\
\end{format}


\section{EMPLOYMENT}

\employer{\textbf{Adobe Systems}}
\title{\emph{Creative Technologies Lab Intern}}
\dates{Summer 2011}
\location{San Francisco, CA}
\begin{position}
Researching creativity support tools for artists and designers.
\end{position}

\employer{\textbf{Pixar Animation Studios}}
\title{\emph{Technical Director Intern}}
\dates{Summer 2009}
\location{Emeryville, CA}
\begin{position}
Developed tools and content for the short ``Day and Night" and other studio films.
\end{position}

\employer{\textbf{Hewlett-Packard}}
\title{\emph{Software Intern}}
\dates{Summer 2008}
\location{Roseville, CA}
\begin{position}
Prototyped an application for print authorization using Microsoft Active Directory.
\end{position}


\section{TEACHING \& MENTORING}

% Going to abuse some commands here...

\employer{\textbf{Teaching Assistant}}
\title{\emph{Stanford CS 348b:}}
\dates{Spring 2014}
\location{Stanford, CA}
\begin{position}
\emph{Image Synthesis Techniques}\\
Held office hours and graded assignments for Stanford's advanced image synthesis course.
\end{position}

\employer{\textbf{Research Mentor}}
\title{\emph{Stanford CURIS Program}}
\dates{Summer 2013}
\location{Stanford, CA}
\begin{position}
I mentored an undergraduate student as part of the Stanford CS Undergraduate Research Internship (CURIS) program. The student, Ben Mildenhall, developed a system for `Bayesian Policy Search,' using Markov Chain Monte Carlo to infer high-reward policies for simple 2D game agents. Ben later contributed to my SIGGRAPH 2015 paper, and he is now a Ph.D. student at UC Berkeley.
\end{position}

\employer{\textbf{Teaching Assistant}}
\title{\emph{Stanford CS 148:}}
\dates{Fall 2011}
\location{Stanford, CA}
\begin{position}
\emph{Introduction to Computer Graphics and Imaging}\\
Held office hours, designed assignments, graded exams, and gave guest lectures for Stanford's introductory graphics course.
\end{position}

\employer{\textbf{Teaching Assistant}}
\title{\emph{UC Berkeley CS 184:}}
\dates{Fall 2009, Spring 2010}
\location{Berkeley, CA}
\begin{position}
\emph{Foundations of Computer Graphics}\\
Led student discussion sections for UC Berkeley's undergraduate graphics course.
\end{position}

\employer{\textbf{Student Facilitator}}
\title{\emph{UCBUGG}}
\dates{December 2008 -- May 2010}
\location{Berkeley, CA}
\begin{position}
Taught fellow students the principles of 3D animation with Maya.
\end{position}

\employer{\textbf{Tutor}}
\title{\emph{UC Berkeley Self-Paced Center}}
\dates{Fall 2008}
\location{Berkeley, CA}
\begin{position}
Tutored students as they learned programming languages at their own pace.
\end{position}


% \section{SERVICE}


\section{OPEN-SOURCE SOFTWARE}

\textbf{WebPPL} \\
\url{http://webppl.org} \\
I am one of the main contributors to WebPPL, a state-of-the-art probabilistic programming language that uses continuation-passing style to support enumeration-based, sampling-based, and variational inference algorithms.
\\ \\
\textbf{probabilistic-js} \\
\url{https://github.com/dritchie/probabilistic-js} \\
I am the author of probabilistic-js, a lightweight embedding of probabilistic programming in Javascript. Like WebPPL, probabilistic-js can run in-browser, and it is the inference engine behind the popular online course \emph{Probabilistic Models of Cognition} (\url{https://probmods.org}).
\\ \\
\textbf{Quicksand} \\
\url{http://dritchie.github.io/quicksand/} \\
I am the author of Quicksand, a lightweight embedding of probabilistic programming in the high-performance Terra language. Quicksand uses Terra's metaprogramming and code-generation constructs to compile fast machine code for MCMC inference on probabilistic programs.


\section{PATENTS}

\textbf{Methods and Apparatus for Comic Creation} (US 20130073952 A1)

\section{AWARDS \& HONORS}
Eurographics Best Paper Honorable Mention, 2015 \\
Stanford Graduate Fellowship, 2010-2015 \\ 
UC Berkeley EECS Departmental Citation, 2010 \\
UC Berkeley Computer Science Highest Achievement Award, 2010 \\
CRA Outstanding Undergraduate Researcher Honorable Mention, 2010 \\
UC Berkeley Edward Frank Kraft Scholarship, 2007 \\

% \section{OTHER ACTIVITIES}

\section{REFERENCES}

\textbf{Pat Hanrahan} \\
\emph{Canon USA Professor of Computer Science} \\
Stanford University  \\
\url{hanrahan@cs.stanford.edu}
\\ \\
\textbf{Noah Goodman} \\
\emph{Assistant Professor of Psychology} \\
Stanford University  \\
\url{ngoodman@stanford.edu}

\end{resume}

\end{document}







