
%----------------------------------------------------------------------------------------
%	PACKAGES AND OTHER DOCUMENT CONFIGURATIONS
%----------------------------------------------------------------------------------------

\documentclass[
11pt, % Main document font size
a4paper, % Paper type, use 'letterpaper' for US Letter paper
oneside, % One page layout (no page indentation)
%twoside, % Two page layout (page indentation for binding and different headers)
headinclude,footinclude, % Extra spacing for the header and footer
BCOR5mm, % Binding correction
]{scrartcl}

\input{structure.tex} % Include the structure.tex file which specified the document structure and layout

\hyphenation{Fortran hy-phen-ation} % Specify custom hyphenation points in words with dashes where you would like hyphenation to occur, or alternatively, don't put any dashes in a word to stop hyphenation altogether

%----------------------------------------------------------------------------------------
%	TITLE AND AUTHOR(S)
%----------------------------------------------------------------------------------------

\title{\normalfont\spacedallcaps{Daniel Ritchie}} % The article title

\author{\spacedallcaps{Research Statement}} % The article author(s) - author affiliations need to be specified in the AUTHOR AFFILIATIONS block

\date{} % An optional date to appear under the author(s)



\begin{document}

\maketitle

My research focuses on the advancement of probabilistic programming languages, particularly for expressing and solving complex procedural modeling and design problems. My work is interdisciplinary, sitting at the intersection of computer graphics, artificial intelligence, and programming languages. I believe in developing general-purpose solutions, rather than specialized fixes for one problem or application domain. This means that on the way to solving computer graphics problems, my work often leads to systems and algorithms that have broader impact for probabilistic programming and probabilistic inference in general.

In procedural modeling, content creators write programs that pseudo-randomly generate 3D graphics content, such as trees, vehicles, or entire environments. This approach to content creation scales to large amounts of diverse content without extensive human effort, and it is becoming more important as the virtual worlds of films, games, and simulations grow larger and more detailed. Traditional procedural modeling tools operate via forward reasoning: the user writes a step-by-step recipe for generating content, such as a 3D model for a house. However, detailed, realistic content that can be interacted with must also satisfy properties that are difficult to guarantee constructively—for example, the house models should be structurally stable. Satisfying such conditions seems to require inverse reasoning, working backwards from the desired result.

Bayesian probabilistic inference provides a well-studied conceptual mechanism for this kind of inversion: the forward recipe serves as a generative prior distribution, and the desired properties can be enforce via a likelihood function. Furthermore, probabilistic programs are a natural choice for representing these types of distributions. They are a Turing-complete, universal probabilistic model, so they can capture the highly structured, recursive, mixed-continuous-discrete nature of many procedural models. They are also human-readable and human-writable, requiring no special expertise in probabilistic inference to author or understand.

Unfortunately, the expressiveness and familiarity of probabilistic programs comes at a cost: inference on such complex programs is often intractable. We must resort to approximate methods that converge slowly, if at all. Thus, most of my work has focused on improving the reliability and runtime efficiency of probabilistic programming inference, with a particular eye towards procedural modeling applications in computer graphics. My research in this space has involved both algorithmic work (improving the statistical efficiency of inference through new or improved algorithms) as well as systems work (making existing inference algorithms more computationally efficient).

\end{document}