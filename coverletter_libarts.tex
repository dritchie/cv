%%%%%%%%%%%%%%%%%%%%%%%%%%%%%%%%%%%%%%%%%
% Long Lined Cover Letter
% LaTeX Template
% Version 1.0 (1/6/13)
%
% This template has been downloaded from:
% http://www.LaTeXTemplates.com
%
% Original author:
% Matthew J. Miller
% http://www.matthewjmiller.net/howtos/customized-cover-letter-scripts/
%
% License:
% CC BY-NC-SA 3.0 (http://creativecommons.org/licenses/by-nc-sa/3.0/)
%
%%%%%%%%%%%%%%%%%%%%%%%%%%%%%%%%%%%%%%%%%

%----------------------------------------------------------------------------------------
%	PACKAGES AND OTHER DOCUMENT CONFIGURATIONS
%----------------------------------------------------------------------------------------

\documentclass[10pt,stdletter,dateno,sigleft]{newlfm} % Extra options: 'sigleft' for a left-aligned signature, 'stdletternofrom' to remove the from address, 'letterpaper' for US letter paper - consult the newlfm class manual for more options

\usepackage{charter} % Use the Charter font for the document text

\newsavebox{\Luiuc}\sbox{\Luiuc}{\parbox[b]{1.75in}{\vspace{0.25in}
\includegraphics[width=\linewidth]{stanford_cs_logo.png}}} % Company/institution logo at the top left of the page
\makeletterhead{Uiuc}{\Lheader{\usebox{\Luiuc}}}

% \newlfmP{sigsize=50pt} % Slightly decrease the height of the signature field
\newlfmP{sigsize=0pt} % Slightly decrease the height of the signature field
\newlfmP{addrfromphone} % Print a phone number under the sender's address
\newlfmP{addrfromemail} % Print an email address under the sender's address
\PhrPhone{Phone} % Customize the "Telephone" text
\PhrEmail{Email} % Customize the "E-mail" text

\lthUiuc % Print the company/institution logo

%URLs
\usepackage{hyperref}

%----------------------------------------------------------------------------------------
%	YOUR NAME AND CONTACT INFORMATION
%----------------------------------------------------------------------------------------

\namefrom{Daniel Ritchie} % Name

\addrfrom{
\today\\[12pt] % Date
736 Escondido Road \#323 \\ % Address
Stanford, CA 94305
}

\phonefrom{(530) 409-6656} % Phone number

\emailfrom{\url{dritchie@stanford.edu}} % Email address

%----------------------------------------------------------------------------------------
%	ADDRESSEE AND GREETING/CLOSING
%----------------------------------------------------------------------------------------

% Commented out b/c I have a separate script that specifies them programmatically.
\newcommand{\Recipient}{Faculty Search Committee}
\newcommand{\University}{Some College}
\newcommand{\UniversityShort}{College}
\newcommand{\Department}{Some Department}
\newcommand{\Address}{123 Pleasant Lane\\City, State 12345}


% \closeline{Sincerely,} % Closing text
\closeline{Sincerely, \newline \newline \includegraphics[width=0.45\linewidth]{signature.jpg} \vspace{-4em}} % Closing text

\greetto{Dear \Recipient,} % Greeting text

\nameto{\Recipient} % Addressee of the letter above the to address

\addrto{
\Department \\
\University \\
\Address
}

%----------------------------------------------------------------------------------------

\begin{document}
\begin{newlfm}

%----------------------------------------------------------------------------------------
%	LETTER CONTENT
%----------------------------------------------------------------------------------------

I am currently a Ph.D. candidate in the Computer Science Department at Stanford University, advised by Pat Hanrahan. I expect to complete my dissertation in June 2015, and I am writing to submit my application for a tenure-track Assistant Professor position in your department.
I find meaning in inspiring and being inspired by bright young adults at the beginning of their academic development, and I would look forward to engaging with such students at \UniversityShort~through research and teaching.

My research focuses on solving procedural modeling and design problems using probabilistic programming.
My work is interdiscplinary, sitting at the intersection of graphics, artificial intelligence, and programming languages. It also provides natural opportunities to engage with students of different expertise and interests. Procedural graphics offers great examples for introductory CS courses on recursion and data structures. Probabilistic programming languages are excellent tools for teaching probabilistic reasoning and Bayesian inference, which are fundamental to machine learning. And the design of these languages involves foundational programming language issues, such as efficient functional programming and incremental computation.

I have been fortunate to have many experiences teaching and mentoring undergraduates in my career thus far. At Stanford and UC Berkeley, I have four times been a teaching assistant for introductory or advanced computer graphics classes. In those roles, I led small student discussion groups, designed assignments, gave lectures, and authored exam questions. During my Ph.D., I have mentored two undergraduates at Stanford. One contributed to a conference paper and is now a graduate student at Berkeley; the other is currently working with me on an ongoing research project.

I am also committed to fostering an inclusive and diverse learning environment and am particularly concerned with closing the gender gap in computer science. I strive to teach accessible classes, to nuture student role models, and to promote awareness of microagressions.
Please see the enclosed diversity statement for more information.
Enclosed also are my curriculum vitae, research and teaching statements, and a list of references (included in the CV). Please do not hesitate to contact me if you require any further information.

%----------------------------------------------------------------------------------------

\end{newlfm}
\end{document}